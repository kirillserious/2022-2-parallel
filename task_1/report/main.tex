\input{formats/diploma.tex}

\usepackage[utf8]{inputenc}                % Кодировка
\usepackage[main=russian, english]{babel}  % Русский язык
\usepackage[pdftex]{graphicx}              % Картинки
\usepackage{indentfirst}                   % Отступ перед абзацами

\begin{document}
    \thispagestyle{empty}
\begin{center}
    \ \vspace{-3cm}

    \includegraphics[width=0.5\textwidth]{title_page/msu.eps}\\

    {\small{\scshape  Московский государственный университет имени~М.~В.~Ломоносова}\\
    Факультет вычислительной математики и кибернетики}

    \vfill

    {\Large Отчет по заданию}

    \vspace{1cm}

    {\LARGE\bfseries <<Численное интегрирование многомерных функций методом~Монте--Карло>>}

    \vspace{1.5cm}

    {\scshape Вариант 2}
\end{center}

\vspace{3cm}

\begin{flushright}
    \large
    \textit{Студент 615 группы}\\
    Егоров Кирилл 
\end{flushright}

\vfill

\begin{center}
    Москва, 2022
\end{center}

\clearpage
    \tableofcontents
    \clearpage
    \section{Постановка задачи}

    В качестве условия задачи выступает следующий фрагмент кода программы на языке~С:
    \begin{verbatim}
    for(i = 2; i <= n+1; ++i)
        C[i] = C[i - 2] + D[i];
    for(i = 2; i <= n+1; ++i)
        for(j = 2; j <= m+1; ++j)
            B[i][j] = B[i][j - 1] + C[n + 1];
    for(i = 2; i <= n+1; ++i){
        A[i][1][1]= C[i];
        for(j = 2; j <= m+1; ++j){
            for(k = 1; k <= n; ++k)
                A[i][j][k] = A[i][j - 1][k - 1] + A[i][j][k];
        }
    }
    \end{verbatim}
    
    В рамках задания необходимо исследовать информационную структуру указанного фрагмента, то есть выявить имеющиеся в ней зависимости по данным и их характер, после чего составить описание информационной структуры на языке разметки Algolang.
    Необходимо привести значения следующих величин для данного алгоритма (в зависимости от параметров программы $n$ и $m$):
    \begin{enumerate}
        \item Число вершин в информационном графе фрагмента (последовательная сложность);
        \item Длина (число дуг) критического пути в информационном графе (параллельная сложность);
        \item Ширина (максимальное число вершин на ярусе) ярусно-параллельной формы (в тексте дайте пояснения, для какой именно ЯПФ приведено значение ширины);
        \item Максимальная глубина вложенности циклов;
        \item Число различных типов дуг (тип дуг определяется направляющим вектором и длиной при фиксированных значениях параметров);
        \item Наличие длинных дуг (т.е. дуг, длина которых зависит от внешних параметров).
    \end{enumerate}

    После исследования информационной структуры требуется разметить параллельные циклы заданного фрагмента программы с использованием директивы OpenMP \texttt{\#pragma omp parallel for}.

    
    \section{Построение информационного графа}
    В рамках задачи был построен информационный граф для значений параметров $n=4$, $m=3$.
    Для этого было написано следующее описание графа на языке Algolang:
    \begin{verbatim}
    <?xml version="1.0"?>
    <algo>
        <params>
            <param name="n" type="int" value="4"></param>
            <param name="m" type="int" value="3"></param>
        </params>
        <block id="0" dims="1">
            <arg name="i" val="2..n+1"></arg>
            <vertex condition="" type="1">
                <in src="i-2"></in>
            </vertex>
        </block>
        <block id="1" dims="2">
            <arg name="i" val="2..n+1"></arg>
            <arg name="j" val="2..m+1"></arg>
            <vertex condition="" type="1">
                <in src="i, j-1"></in>
                <in bsrc="0" src="n+1"></in>
            </vertex>
        </block>
        <block id="2" dims="3">
            <arg name="i" val="2..n+1"></arg>
            <arg name="j" val="1..m+1"></arg>
            <arg name="k" val="1..n"></arg>
            <vertex condition="(j==1) and (k==1)" type="1">
                <in bsrc="0" src="i"></in>
            </vertex>
            <vertex condition="(j>1)" type="1">
                <in src="i, j-1, k-1"></in>
                <in src="i, j, k"></in>
            </vertex>
        </block>
    </algo>
    \end{verbatim}

    Данное описание было построено в системе Algoload.
    Ниже представлена визуализация в проекциях на плоскости $XY$, $YZ$, $XZ$, а также на плоскость, с которой субъективно лучше всего видно структуру информационного графа рассматриваемого алгоритма.
\end{document}